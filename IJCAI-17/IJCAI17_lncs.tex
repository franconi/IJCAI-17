\documentclass[envcountsame,draft]{llncs}

%\usepackage{MnSymbol}
%\usepackage{times}
\usepackage{latexsym,amssymb,amsmath,amssymb}
\usepackage{xspace}
\usepackage{graphicx}
\usepackage{misc}
\usepackage{mathabx}
\usepackage{color} 
\usepackage{named} 
%\usepackage{pifont}
\usepackage{mathrsfs}
\usepackage{tikz}
\usetikzlibrary{graphs}
\usepackage[obeyDraft]{todonotes}

\usepackage{stackrel}
\usepackage{relsize}

\newcommand{\UMLfull}{UML\ensuremath{_\textit{full}}\xspace}
\newcommand{\UMLbool}{UML\ensuremath{_\textit{bool}}\xspace}
\newcommand{\UMLref}{UML\ensuremath{_\textit{ref}}\xspace}

\newcommand{\UCD}{UCD\xspace}
\newcommand{\UCDs}{UCDs\xspace}

%\newcommand{\exptime}{\textsc{ExpTime}\xspace}
\newcommand{\ptime}{\textsc{P}\xspace}
\newcommand{\pspace}{\textsc{PSpace}\xspace}
\newcommand{\nlogspace}{\textsc{NLogSpace}\xspace}
\newcommand{\np}{\textsc{NP}\xspace}

\newcommand{\isa}{\textsc{isa}\xspace}

\newcommand{\nb}[1]{\textcolor{red}{\textdagger}\marginpar{\scriptsize\raggedright\textcolor{red}{#1}}}

\newcommand{\greif}[1]{\ensuremath{\bigocirc#1}}
\newcommand{\lreif}[1]{\ensuremath{\bigodot#1}}

\newcommand\restr[2]{{% we make the whole thing an ordinary symbol
  \left.\kern-\nulldelimiterspace % automatically resize the bar with \right
  #1 % the function
  \vphantom{\big|} % pretend it's a little taller at normal size
  \right|_{#2} % this is the delimiter
  }}
  
\newcommand{\Int}[1]{#1^{\Imc}\xspace}
%\renewcommand\dom{\ensuremath{\Delta^{\Imc}}\xspace}
\renewcommand\dom{\ensuremath{\Delta}\xspace}
\newcommand{\KB}{\ensuremath{\mathcal{KB}}\xspace}
\newcommand{\per}{\mathpunct{\mbox{\bf .}}}
\newcommand{\pth}[2]{\ensuremath{\textsc{path}_{\mathscr{T}}(#1,#2)}\xspace}
\newcommand{\chd}[2]{\ensuremath{\textsc{child}_{\mathscr{T}}(#1,#2)}\xspace}
\newcommand{\A}{\ensuremath{\mathcal{A}}\xspace}
\newcommand{\Ob}{\ensuremath{\mathcal{O}}\xspace}
%\newcommand{\KB}{\ensuremath{\matcal KB}\xspace}

\DeclareMathOperator*{\RAJoin}{\Join}

%\renewcommand{\baselinestretch}{0.991}

%%% Color definitions (taken from rgb.tex)
\definecolor{black}           {rgb}{0.00,0.00,0.00}
\definecolor{blue}            {rgb}{0.00,0.00,1.00}
\definecolor{midnightblue}    {rgb}{0.10,0.10,0.44}
\definecolor{firebrick}       {rgb}{0.70,0.13,0.13}
\definecolor{forestgreen}     {rgb}{0.13,0.55,0.13}
\definecolor{gray}            {rgb}{0.75,0.75,0.75}
\definecolor{gray50}          {rgb}{0.50,0.50,0.50}
\definecolor{magenta}         {rgb}{1.00,0.00,1.00}
\definecolor{mediumorchid}    {rgb}{0.73,0.33,0.83}
\definecolor{mediumslateblue} {rgb}{0.48,0.41,0.93}
\definecolor{orange}          {rgb}{1.00,0.65,0.00}
\definecolor{darkorange}      {rgb}{1.00,0.55,0.00}
\definecolor{orangered}       {rgb}{1.00,0.27,0.00}
\definecolor{myorange}        {rgb}{1.00,0.40,0.00}
\definecolor{red}             {rgb}{1.00,0.00,0.00}
\definecolor{sienna}          {rgb}{0.63,0.32,0.18}
\definecolor{violetred}       {rgb}{0.82,0.13,0.56}
\definecolor{white}           {rgb}{1.00,1.00,1.00}
\definecolor{violetblue}      {rgb}{0.40,0.20,1.00}
\definecolor{brown}           {rgb}{0.65,0.16,0.16}

\definecolor{myblue}{HTML}{106872}% 007777}%8888}
\definecolor{mygray}{HTML}{CCCCCC}%6633CC}
\definecolor{mygrey}{HTML}{cc8879}%6633CC}
\definecolor{mydarkgray}{HTML}{aaaaaa}
\definecolor{myorange}{HTML}{cc6877}%995533}%

% #1 - width
% #2 - color
% #3 - text
\newcommand{\coloredboxw}[3]{
  \vspace{0.3cm}
  \begin{tikzpicture}
    \draw let \p1=($(#1,0)-(1,0)$) in node[draw, rectangle, color=#2,
    fill=#2!10!, text=black, rounded corners=10, inner xsep=0.5cm,inner ysep=1mm] {
      \begin{minipage}[t]{\x1}
        #3
      \end{minipage}};
  \end{tikzpicture}
  \vspace{-0.5cm}
}

% #1 - color
% #2 - text
\newcommand{\coloredbox}[2]{
  \coloredboxw{\linewidth}{#1}{#2}
}

%%%%%%%% Colors
\newcommand {\bblu}[1]  {\color{blue}{\sf #1}}
\newcommand {\bred}[1]  {\color{red}\sf{#1}}
\newcommand {\bgreen}[1]  {\color{green}\sf{#1}}

\newcommand{\bb}[1]{\textcolor{blue}{#1}}
\newcommand{\bbd}[1]{\textcolor{midnightblue}{#1}}
\newcommand{\rr}[1]{\textcolor{red}{#1}}
\newcommand{\mm}[1]{\textcolor{magenta}{#1}}
\newcommand{\nn}[1]{\textcolor{black}{#1}}
\renewcommand{\gg}[1]{\textcolor{forestgreen}{#1}}
\newcommand{\gr}[1]{\textcolor{gray50}{#1}}
\renewcommand{\ss}[1]{\textcolor{sienna}{#1}}
\newcommand{\oo}[1]{\textcolor{myorange}{#1}}
\newcommand{\vv}[1]{\textcolor{violetred}{#1}}
\newcommand{\ww}[1]{\textcolor{white}{#1}}
\newcommand{\bl}[1]{\textcolor{mediumslateblue}{#1}}
\newcommand{\vb}[1]{\textcolor{violetblue}{#1}}
\newcommand{\fb}[1]{\textcolor{firebrick}{#1}}

%%%%%%%%%%%%%%%%%%%%%%%%%%%%%%%%%%%%%%%%%%%%%%%%%%%%%%%%%%%%%%%%%%%%%%
%\pdfinfo{
%	/Title (The n-ary DLR description logics extended with Labelled Tuples, Projections, Functional Dependencies and Objectification)
%	/Title (The very expressive n-ary DLR+- description logic)
%	/Author (Alessandro Artale, Enrico Franconi, Rafael Penaloza)}

%\title{The \textit{n}-ary \DLR description logic extended with\\ Labelled Tuples, Projections, Functional Dependencies, Identification Constraints, and Objectification}
\title{The very expressive \textit{n}-ary description logic \DLRpm}

\author{Alessandro Artale, Enrico Franconi, Rafael Pe\~naloza, Francesco Sportelli}
\institute{KRDB Research Centre, 
Free University of Bozen-Bolzano, Italy\\
\texttt{\{artale,franconi,penaloza,sportelli\}@inf.unibz.it}
}

\begin{document}

\date{}
\maketitle

%%%%%%%%%%%%%%%%%%%%%%%%%%%%%%%%%%%%%%%%%%%%%%%%%%%%%%%%%%%%%%%%%%%%%%

\begin{abstract}
  We introduce the logic \DLRp an extension of the n-ary description
  logic \DLR to deal with attribute-labelled tuples (generalising the
  positional notation), with arbitrary projections of relations
  (inclusion dependencies), generic functional dependencies,
  identification constraints\nb{A: added} and with global and local
  objectification (reifying relations, or their projections,
  introducing a global or a local identifier, respectively\nb{A:
    check}).  The\nb{A: added} logic is equipped with both TBox and
  ABox axioms forming a \DLRp knowledge base.  We show how a simple
  syntactic condition on the appearance of projections and functional
  dependencies in a knowledge base makes the language decidable
  without increasing the computational complexity of the basic \DLR
  language.
\end{abstract}

%%%%%%%%%%%%%%%%%%%%%%%%%%%%%%%%%%%%%%%%%%%%%%%%%%%%%%%%%%%%%%%%%%%%%%
%%%%%%%%%%%%%%%%%%%%%%%%%%%%%%%%%%%%%%%%%%%%%%%%%%%%%%%%%%%%%%%%%%%%%%
\section{Introduction}

We introduce in this paper the language \DLRp which extends the $n$-ary description logics \DLR~\cite{calvanese:et:al:98b,BCMNP03} and \DLRID~\cite{CalvaneseGL01} as follows:

\begin{itemize}
\item the semantics is based on attribute-labelled tuples: an element of a tuple is identified by an attribute and not by its position in the tuple, e.g.,
the relation \texttt{Person} has attributes \texttt{firstname}, \texttt{lastname}, \texttt{age}, \texttt{height} with instance:\\
\texttt{$\langle$ firstname: Enrico, lastname: Franconi, age: 53, height: 1.90$\rangle$};
\item renaming of attributes is possible, e.g., to recover the positional semantics:\\ $\texttt{firstname,lastname,age,height}\rightleftarrows \texttt{1,2,3,4}$;
\item it can express projections of relations, and therefore inclusion dependencies, e.g.,
$\ATLEASTRS{\texttt{firstname,lastname}}\texttt{Student}\sqsubseteq\ATLEASTRS{\texttt{firstname,lastname}}\texttt{Person}$;
\item it can express multiple-attribute cardinalities, and therefore functional dependencies and multiple-attribute keys, e.g., the functional dependency from \texttt{firstname,} \texttt{lastname} to \texttt{age} in \texttt{Person} can be written as:\\
$\exists[\texttt{firstname,lastname}] \texttt{Person} \sqsubseteq$ 

\hspace{2em} $\exists^{\leq 1}[\texttt{firstname,lastname}](\exists[\texttt{firstname,lastname,age}] \texttt{Person})$;
\item it can express global and local objectification (also known as reification): a tuple may be identified  by a unique global identifier, or by an identifier which is unique only within the interpretation of a relation, e.g., to identify the name of a person we can write 
$\texttt{Name}\sqsubseteq\lreif{\exists[\texttt{firstname,lastname}] \texttt{Person}} $.
\end{itemize}

We show how a simple syntactic condition on the appearance of projections in the knowledge base makes the language decidable without increasing the computational complexity of the basic \DLR language. We call \DLRpm this fragment of \DLRp. \DLRpm is able to correctly express the UML fragment as introduced in~\cite{BeCD05-AIJ-2005,ACKRZ:er07} and the ORM fragment as introduced in~\cite{DBLP:conf/otm/FranconiM13}.

%%%%%%%%%%%%%%%%%%%%%%%%%%%%%%%%%%%%%%%%%%%%%%%%%%%%%%%%%%%%%%%%%%%%%%
%%%%%%%%%%%%%%%%%%%%%%%%%%%%%%%%%%%%%%%%%%%%%%%%%%%%%%%%%%%%%%%%%%%%%%

\section{The Description Logic \DLRp}
\label{sec:syntax}

We first define the syntax of the language \DLRp. A \DLRp \emph{signature}
is a tuple
$\mathcal{L}=(\mathcal{C},\mathcal{R},\mathcal{O},\mathcal{U},\tau)$ 
where $\mathcal{C}$, $\mathcal{R}$, $\mathcal{O}$ and $\mathcal{U}$ are finite, 
mutually disjoint sets of 
\emph{concept names}, \emph{relation names}, \emph{individual names}, and \emph{attributes}, respectively,
and $\tau$ is a \emph{relation signature} function, associating a set of attributes to
each relation name
$\tau(R\!N)=\{U_1,\ldots,U_n\}\subseteq \Umc$ with $n\geq 2$.
The \emph{arity} of a relation $R$ is the number of the attributes in its signature; i.e., 
$\textsc{arity}(R)=\left|\tau(R)\right|$.
%
The syntax of concepts $C$, relations $R$, formulas $\varphi$, and
attribute renaming axioms $\vartheta$ is given in
Figure~\ref{fig:dlrp}, where $C\!N\in\mathcal{C}$, $R\!N\in\mathcal{R}$, 
$U\in\mathcal{U}$, 
$q$ is a positive integer and $2\leq k < \textsc{arity}(R)$. 
%
\begin{figure*}
	[t] 
	\begin{center}
		\renewcommand{\arraystretch}{1.2} $
		\begin{array}{r@{\hspace{2ex}}c@{\hspace{2ex}}l} 
			C & \to & \top\ \mid\ \bot\ \mid\ C\!N\ \mid\ \neg C\ \mid\ C_{1}\sqcap C_{2}\ \mid\ C_{1}\sqcup C_{2}\ \mid\ \EXISTR{q}{U_i} R\ \mid\ \greif{R}\ \mid\ \lreif{R\!N}\\
			%
			R & \to & R\!N\ \mid\ R_1\setminus R_2\ \mid\ R_{1}\sqcap R_{2}\mid\ R_{1}\sqcup R_{2}\mid\ \selects{U_i}{C}{R}\ \mid\ \EXISTR{q}{U_1,\ldots,U_k} R\\
			\varphi & \to & C_1\sqsubseteq C_2\ \mid\ R_1\sqsubseteq R_2 \mid C\!N(o) \mid R\!N(U_1\!:\!o_1,\ldots,U_n\!:\!o_n) \mid o_1 = o_2 \mid o_1 \neq o_2 \\
			\vartheta & \to & U_1 \rightleftarrows U_2
		\end{array}
		$ 
		\renewcommand{\arraystretch}{1} 
	\end{center}
	\caption{\label{fig:dlrp} Syntax of \DLRp.} 
\end{figure*}
%
We extend the signature function $\tau$ to arbitrary relations as specified in Figure~\ref{fig:syn:tau}.
%
\begin{figure*}[t] 
	\begin{center}
		\renewcommand{\arraystretch}{1.2} $ { 
		\begin{array}{r@{\hspace{1ex}}l@{\hspace{3ex}}l@{\hspace{.3ex}}} 
			\tau(R_1\setminus R_2) = & \tau(R_1) & \text{if } \tau(R_1)=\tau(R_2)\\
			\tau(R_{1}\sqcap R_{2}) = & \tau(R_1) & \text{if } \tau(R_1)=\tau(R_2)\\
			\tau(R_{1}\sqcup R_{2}) = & \tau(R_1) & \text{if } \tau(R_1)=\tau(R_2)\\
			\tau(\selects{U_i}{C}{R}) = & \tau(R) & \text{if } U_i\in\tau(R)\\
			\tau(\EXISTR{q}{U_1,\ldots,U_k} R) = & \{U_1,\ldots,U_k\} & \text{if } \{U_1,\ldots,U_k\}\subset \tau(R)\\
			\tau(R) = & \emptyset & \text{otherwise} 
		\end{array}
		}$ 
		\renewcommand{\arraystretch}{1} 
	\end{center}
	\caption{\label{fig:syn:tau} The signature of \DLRp relations.} 
\end{figure*}

A \DLRp \emph{TBox} \Tmc is a finite set of \emph{concept inclusion} axioms of the form $C_1\sqsubseteq C_2$ 
and \emph{relation inclusion} axioms of the form $R_1\sqsubseteq R_2$.
We will often use $X_1\equiv X_2$ as a shortcut for the two axioms 
$X_1\sqsubseteq X_2$ and $X_2\sqsubseteq X_1$. 
A \DLRp \emph{ABox} \Amc is a finite set of \emph{concept instance} axioms of the form $C\!N(o)$, 
\emph{relation instance} axioms of the form $R\!N(U_1\!:\!o_1,\ldots,U_n\!:\!o_n)$, and 
\emph{same/distinct individual} axioms of the form $o_1 = o_2$ and $o_1 \neq o_2$, with $o_i\in\Ob$. 
It is easy to see that restricting ABox axioms to concept and relation names only does not affect the 
expressivity of \DLRp due to the availability of TBox axioms.
%
A set of renaming axioms forms a \emph{renaming schema}, which induces an equivalence relation 
$(\rightleftarrows,\mathcal{U})$ over the attributes $\mathcal{U}$, providing a partition of $\mathcal{U}$ into equivalence 
classes each one representing the alternative ways to name attributes. We write $[U]_\Re$ to denote the equivalence 
class of the attribute $U$ w.r.t. the equivalence relation $(\rightleftarrows,\mathcal{U})$. 
We allow only \emph{well founded} renaming schemas, namely schemas such that each equivalence class $[U]_\Re$ in 
the induced equivalence relation never contains two attributes from the same relation signature. 
%
We use the shortcut $U_1\ldots U_n\rightleftarrows U'_1\ldots U'_n$ to group many renaming axioms 
with the obvious meaning that $U_i\rightleftarrows U'_i$, for all $i=1,\ldots, n$.
%
A \DLRp knowledge base (KB) $\mathcal{KB}=(\Tmc\!,\Amc,\Re)$ is composed by a TBox \Tmc, an ABox \Amc, and 
a renaming schema $\Re$.

%The selection expression $\selects{U_i}{C}{R\!N}$ denotes the
%relation $R\!N$ where the attribute $U_i$ is restricted to the
%concept $C$.
%%
%Global and local \emph{objectification} (also known as reification) of
%relations are denoted as $\greif{R}, \lreif{R\!N}$,
%respectively.
%%
%The \emph{unary projection} expression, $\EXISTR{q}{U_i} R$, denotes a
%concept expression as a generalisation with cardinalities of the unary
%projection operator over the attribute $U_i$ of the relation $R$; we
%denote with $\exists[U_i]R$ the plain unary projection. In \DLRp we
%can express $n$-ary projections (possibly with cardinalities) denoted
%as $\EXISTR{q}{U_1,\ldots,U_k} R$.
%%
%% With $\rho_{U_1/U'_1,\ldots,U_l/U'_l}R$ we denote the relation $R$
%% where attributes $\{U_1,\ldots,U_l\}$ are renamed as
%% $\{U'_1,\ldots,U'_l\}$. 

The renaming schema reconciles the attribute and the positional perspectives on relations (see also the similar 
perspectives in relational databases~\cite{AbiteboulHV95}). They are crucial when expressing both inclusion axioms and 
operators ($\sqcap,~\sqcup,~\setminus$) between relations, which make sense only over \emph{union compatible} 
relations. Two relations $R_1,R_2$ are union compatible if their signatures are equal up to the attribute renaming 
induced by the renaming schema $\Re$, namely, $\tau(R_1)=\{U_1,\ldots,U_n\}$ and $\tau(R_2)=\{V_1,\ldots,V_n\}$ 
have the same arity $n$ and $[U_i]_\Re=[V_i]_\Re$ for each $1\leq i\leq n$. Notice that through the renaming schema, 
relations can use just local attribute names that can then be renamed when composing relations.
Also note that it is obviously possible for the same attribute to appear in the signature of different relations.

%\vspace{2ex}
%
%To show the expressive power of the language, let us consider the following example with tree relation names $R_1, R_2$ and $R_3$ with the following signature:
%%
%\begin{align*}
%  \tau(R_1)  &= \{U_1,U_2,U_3,U_4,U_5\}\\
%  \tau(R_2)  &= \{V_1,V_2,V_3,V_4,V_5\}\\
%  \tau(R_3)  &= \{W_1,W_2,W_3,W_4\}
%\end{align*}
%%
%To state that $\{U_1,U_2\}$ is the \emph{multi-attribute key} of $R_1$ we add the axiom:
%  \begin{align*}
%    \exists[U_1,U_2] R_1 \sqsubseteq \exists^{\leq 1}[U_1,U_2] R_1
%  \end{align*}
%%
%where $\exists[U_1,\ldots,U_k] R$ stands for $\exists^{\geq 1}[U_1,\ldots,U_k] R$. To express that there is a \emph{functional dependency} from the attributes $\{V_3,V_4\}$ to the attribute $\{V_5\}$ of $R_2$ we add the axiom:
%\begin{align}\label{funct-dep}
%      \exists[V_3,V_4] R_2 \sqsubseteq \exists^{\leq 1}[V_3,V_4](\exists[V_3,V_4,V_5] R_2)
%\end{align}
%%
%The following axioms express that $R_2$ is a sub-relation of $R_1$ and
%that a projection of $R_3$ is a sub-relation of a projection of $R_1$,
%together with the corresponding axioms for the
%renaming schema to explicitly specify the % exact behaviour
%correspondences between the attributes of the two inclusion dependencies:
%%2
%\begin{align*}
%  R_2 &\sqsubseteq R_1\\
%  \exists[W_1,W_2,W_3] R_3 &\sqsubseteq \exists[U_3,U_4,U_5] R_1\\
%  V_1V_2V_3V_4V_5 &\rightleftarrows U_1U_2U_3U_4U_5 \\
%  W_1W_2W_3 &\rightleftarrows U_3U_4U_5
%\end{align*}
\begin{example}
\label{exa:basic}
\todo{{\bf rpn:} I simplified the example a bit, but can roll back to previous example if needed}
Consider the relation names $R_1, R_2$ s.t.\
%
%\begin{align*}
  $\tau(R_1)  = \{W_1,W_2,W_3,W_4\}$,
%  $\tau(R_1)  = \{U_1,U_2,U_3,U_4,U_5\}$,
  $\tau(R_2)  = \{V_1,V_2,V_3,V_4,V_5\}$.
%\end{align*}
%
The axiom:
  \begin{align*}
    \exists[W_1,W_2] R_1 \sqsubseteq \exists^{\leq 1}[W_1,W_2] R_1
  \end{align*}
states that $\{W_1,W_2\}$ is the \emph{multi-attribute key} of $R_1$.%
\footnote{$\exists[U_1,\ldots,U_k] R$ stands for $\exists^{\geq 1}[U_1,\ldots,U_k] R$.} 
It is also possible to express that there is a \emph{functional dependency} from the 
attributes $\{V_3,V_4\}$ to $\{V_5\}$ of $R_2$ through the axiom:
\begin{align}\label{funct-dep}
      \exists[V_3,V_4] R_2 \sqsubseteq \exists^{\leq 1}[V_3,V_4](\exists[V_3,V_4,V_5] R_2).
\end{align}
%
Relationships between projections of relations can be expressed as follows.
Consider the attribute renaming axiom
$W_1W_2W_3 \rightleftarrows V_3V_4V_5$.
Then, one can express 
that a projection of $R_1$ is a sub-relation of a projection of $R_2$
by the axiom
\begin{align*}
%  R_2 &\sqsubseteq R_1\\
  \exists[W_1,W_2,W_3] R_1 &\sqsubseteq \exists[V_3,V_4,V_5] R_2.
%  V_1V_2V_3V_4V_5 &\rightleftarrows U_1U_2U_3U_4U_5 \\
%  W_1W_2W_3 &\rightleftarrows U_3U_4U_5
\end{align*}
\end{example}

%%%%%%%%%%%%%%%%%%%%%%%%%%%%%%%%%%%%%%%%%%%%%%%%%%%%%%%%%%%%%%%%%%%%%%
%%%%%%%%%%%%%%%%%%%%%%%%%%%%%%%%%%%%%%%%%%%%%%%%%%%%%%%%%%%%%%%%%%%%%%

%\section{Semantics}  

\begin{figure}[t] 
	\centering
%		\renewcommand{\arraystretch}{1.2} $ { 
%		\begin{array}{r@{\hspace{1ex}}l@{}} 
		\begin{align*}
			\Int{\top} = {} & \dom\\
			%
			\Int{\bot} = {} & \emptyset\\
			%
			\Int{(\neg C)} = {} & \Int{\top} \setminus \Int{C}\\
			%
			\Int{(C_{1}\sqcap C_{2})} = {} & \Int C_{1} \cap \Int C_{2}\\
			%
			\Int{(C_{1}\sqcup C_{2})} = {} & \Int C_{1} \cup \Int C_{2}\\
			%
			\Int{(\EXISTR{q}{U_i} R)} = {} & \{d\in\dom\mid~\left|\{t\in\Int R\mid t[\rho(U_i)]=d\}\right| \lesseqgtr q \}\\
			%
			\Int{(\greif{R})} = {} & \{d\in \dom \mid d=\imath(t) \land t\in \Int{R}\}\\
			%
			%
%			\vspace{2ex}
			%
			%
			\Int{(\lreif{R\!N})} = {} & \{d\in \dom \mid d=\ell_{R\!N}(t)\land t\in \Int{R\!N}\}\\
			%
			\Int{(R_1\setminus R_2)} = {} & \Int R_{1} \setminus \Int R_{2}\\
			%
			\Int{(R_{1}\sqcap R_{2})} = {} & \Int R_{1} \cap \Int R_{2}\\
			%
			\Int{(R_{1}\sqcup R_{2})} = {} & \{t\in\Int R_{1}\cup\Int R_{2}\mid \rho(\tau(R_1))= \rho(\tau(R_2))\}\\
			%
			\Int{(\selects {U_i}{C}{R})} = {} & \{t\in\Int{R} \mid t[\rho(U_i)]\in\Int{C}\}  \\
			%
			\Int{(\EXISTR{q}{U_1,\ldots,U_k} R)} = {} & 
					\{ \langle \rho(U_1):d_1,\ldots,\rho(U_k):d_k\rangle \in T_{\dom}(\{\rho(U_1),\ldots,\rho(U_k)\}) \mid {}  \\ & \ 
			\left. \left|\{t\in\Int R \mid t[\rho(U_1)]=d_1,\ldots,t[\rho(U_k)]=d_k\}\right| \lesseqgtr q \right\}
%		\end{array}
		\end{align*}
%		}$ 
%		\renewcommand{\arraystretch}{1} 
%	\end{center}
	\caption{\label{fig:sem:dlrp} Semantics of \DLRp expressions.} 
\end{figure}

%\todo{{\bf rpn:} removed the division of syntax and semantics to make just one section. Can be rolled back if needed}
The semantics of \DLRp uses of the notion of \emph{labelled tuples} over a domain $\Delta$: a 
\emph{$\mathcal{U}$-labelled tuple over $\Delta$} (or \emph{tuple} for short) is a function 
$t \colon \mathcal{U} \to \Delta$. For $U\in \mathcal{U}$, we write $t[U]$ to refer to the domain element ${d\in \Delta}$ 
labelled by $U$, if the function $t$ is defined for $U$---that is, if the attribute $U$ is a label of the tuple $t$. 
Given $d_1,\dots,d_n\in \Delta$, the expression ${\langle U_1\colon d_1,\ldots,U_n\colon d_n\rangle}$ stands for the 
tuple $t$ such that ${t[U_i]=d_i}$, for ${1\leq 1\leq n}$. 
% and the expression ${{U_i:d_i}\in t}$ stands for ${t[U_i]=d_i}$, for ${1\leq 1\leq n}$.
The \emph{projection} of the tuple $t$ over the attributes ${U_1,\ldots,U_k}$ (i.e., the function $t$ 
restricted to be undefined for the labels not in ${U_1,\ldots,U_k}$)
is denoted by ${t[U_1,\ldots,U_k]}$. 
The\nb{A: added} relation signature function $\tau$ can be applied also to labelled
tuples to obtain the set of labels on wich the tuple is defined.
$T_\Delta(\mathcal{U})$ denotes the set of all $\mathcal{U}$-labelled tuples over $\Delta$.

A \DLRp \emph{interpretation}, $\Imc = (\dom, \cdot^\Imc, \rho, \imath, L)$ consisting of a nonempty 
\emph{domain} $\dom$, an \emph{interpretation function} $\cdot^\Imc$, a \emph{renaming function} $\rho$, a \emph{global objectification function} $\imath$, and a family $L$ containing one \emph{local objectification function} 
$\ell_{R\!N_i}$ for each named relation $R\!N_i\in\mathcal{R}$.
%
The renaming function $\rho$ is a total function ${\rho:\mathcal{U}\to\mathcal{U}}$ representing a canonical renaming 
for all attributes. For brevity, we denote $\rho(\{U_1,\ldots,U_k\}) = \{\rho(U_1),\ldots,\rho(U_k)\}$.
%
The global objectification function is an injective function, ${\imath:T_{\dom}(\Umc) \to \dom}$, associating a 
\emph{unique} global identifier to each possible tuple.
%
% and\nb{A: dobbiamo re-introdurre} it is such that $\imath(\langle U:d \rangle)=d$, for any
% $U\in\Umc$ and $d\in\dom$, namely it identifies unary tuples with
% their value.
The local objectification functions, ${\ell_{R\!N_i}:T_{\dom}(\Umc) \to \dom}$, are distinct for each relation name in the 
signature, and as the global objectification function they are injective: they associate an identifier---which is unique only 
within the interpretation of a relation name---to each possible tuple.
%and they are such that $\ell_{R\!N_i}(\langle U:d \rangle)=d$ for any $U\in\Umc$ and $d\in\dom$.
The interpretation function $\cdot^\Imc$ assigns a domain element to
each individual, $o^{\mathcal{I}}\in\dom$ (notice\nb{A: added} that UNA is not enforced), a set of 
domain elements to each concept name, $C\!N^{\mathcal{I}}\subseteq \dom$, and a set of $\Umc$-labelled tuples 
over $\dom$ to each relation name conforming with its signature and the renaming function
$R\!N^{\mathcal{I}}\subseteq T_{\dom}(\{\rho(U)\mid U\in\tau(R\!N)\}).$ 
%
This function $\cdot^\Imc$ is unambiguously extended over concept and relation expressions as specified in 
Figure~\ref{fig:sem:dlrp}.

The interpretation $\Imc$ satisfies the concept inclusion axiom
$C_1\sqsubseteq C_2$ if $\Int C_1\subseteq \Int C_2$, it satisfies the
relation inclusion axiom $R_1\sqsubseteq R_2$ if
$\Int R_1\subseteq \Int R_2$, it satisfies a concept instance axiom
$C\!N(o)$ if $\Int o\in \Int {C\!N}$, it satisfies a relation instance
axiom $R\!N(U_1\!:\!o_1,\ldots,U_n\!:\!o_n)$ if
${\langle
  \rho(U_1)\colon\Int{o_1},\ldots,\rho(U_n)\colon\Int{o_n}\rangle}\in
\Int {R\!N}$, it satisfies a same individuals axiom $o_1 = o_2$ if
$\Int{o_1} = \Int{o_2}$, and it satisfies a distinct individuals axiom
$o_1 \neq o_2$ if $\Int{o_1} \neq \Int{o_2}$.  $\Imc$ satisfies a
renaming schema $\Re$ if
%the renaming function $\rho$ renames the attributes in a consistent way with respect to $\Re$, namely if
%$$\forall U\per\rho(U)\in[U]_\Re\land\forall V\in [U]_\Re\per\rho(U)=\rho(V).$$
for all $U,V\in\mathcal{U}$, (i) $\rho(U)\in[U]_\Re$, and (ii) if $V\in [U]_\Re$, then $\rho(U)=\rho(V)$.
\Imc is a \emph{model} for a knowledge base $(\mathcal{T}\!,\mathcal{A},\Re)$ if it satisfies all the axioms in the 
TBox $\mathcal{T}$ and in the ABox $\mathcal{A}$, and the renaming schema $\Re$. 

\emph{KB satisfiability} refers to the problem of deciding the existence of a model of a given knowledge base;
\emph{concept satisfiability} (resp. \emph{relation satisfiability}) is the problem of deciding whether there is a model of 
the knowledge base that assigns a non-empty extension to a given concept (resp. relation); and 
\emph{entailment} is to check whether a given knowledge base logically implies an axiom, that is, whenever all the models of the knowledge base are also models of the axiom.
%
For instance, let $\mathcal{KB}$ be the KB containing all the axioms in Example~\ref{exa:basic}. Then
$V_3,V_4$ are a key for the relation $R_2$; that is,
%
\[
  \mathcal{KB}\models \exists[V_3,V_4] R_2 \sqsubseteq \exists^{\leq 1}[V_3,V_4] R_2.
\]
%

%%%%%%%%%%%%%%%%%%%%%%%%%%%%%%%%%%%%%%%%%%%%%%%%%%%%%%%%%%%%%%%%%%%%%%%%%%

\section{Expressiveness of \DLRp}

\DLRp is a very expressive DL capable of asserting several kinds of constraints that are important in the context of relational databases, such as \emph{equijoins}, \emph{functional dependency} axioms, \emph{key} axioms, \emph{external uniqueness} axioms, and \emph{identification} axioms.

An \emph{equijoin} among two relations with disjoint signatures is the set of all combinations of tuples in the relations that are equal on their selected attribute names. Given two relations $R_1$ and $R_2$ with $\tau(R_1)=\{U^1,U^1_1,\ldots,U^1_{n_1}\}$ and $\tau(R_2)=\{U^2,U^2_1,\ldots,U^2_{n_2}\}$ their equijoin over $U^1$ and $U^2$ is the relation  $R = R_1\!\stackrel[U^1=U^2]{}{\mathlarger{\mathlarger{\mathlarger{\bowtie}}}}\!R_2$ with the signature $\tau(R)=\{U^1,U^1_1,\ldots,U^1_{n_1},U^2_1,\ldots,U^2_{n_2}\}$, which can be expressed in \DLRp with the axioms:

\vspace{1ex}
$
\exists[U^1\!,U^1_1,\ldots,U^1_{n_1}]R\equiv R_1 \sqcap (\selects{U^1}{(\exists[U^1]R_1\sqcap \exists[U^2]R_2)}{R_1})$

$
\exists[U^2\!,U^2_1,\ldots,U^2_{n_2}]R\equiv R_2 \sqcap (\selects{U^2}{(\exists[U^1]R_1\sqcap \exists[U^2]R_2)}{R_2}) \hspace{2em}U^2\rightleftarrows U^1
$.

\vspace{1ex}
A \emph{functional dependency} axiom $(R:U_1\ldots U_j \rightarrow U)$ states that the values of the attributes $U_1\ldots U_j$ uniquely determine the value of the attribute $U$ in the relation $R$. Formally, an interpretation $\Imc $ satisfies the functional dependency axiom $(R:U_1\ldots U_j \rightarrow U)$ if, for all $s,t\in R^\Imc$, $s[U_1] = t[U_1], \ldots, s[U_j] = t[U_j]$ imply $s[U] = t[U]$. Functional dependency axioms are also called \emph{internal uniqueness} axioms~\cite{halpin2008}. A functional dependency can be expressed in \DLRp, assuming that $\{U_1,\ldots, U_j, U\}\subseteq \tau(R)$, with the axiom:
%
$$\exists[U_1,\ldots, U_j]R\sqsubseteq\exists^{\leq 1}[U_1,\ldots, U_j](\exists[U_1,\ldots, U_j, U] R).$$

A special case of functional dependency axiom is the \emph{key} axiom $(R:U_1\ldots U_j \rightarrow R)$, which states that the values of the key attributes $U_1\ldots U_j$ of a relation $R$ uniquely identify tuples in the relation itself. A key axiom can be expressed in \DLRp, assuming that $\{U_1\ldots U_j\}\subseteq\tau(R)$, with the axiom:
%
$$\exists[U_1,\ldots, U_j]R\sqsubseteq\exists^{\leq 1}[U_1,\ldots, U_j]R.$$

The \emph{external uniqueness} axiom $([U^1]R_1\downarrow\ldots \downarrow [U^h]R_h)$ states that the join $R$ of the relations $R_1,\ldots,R_h$ via the attributes $U^1,\ldots,U^h$ has the joined attribute functionally dependent on all the others~\cite{halpin2008}. This can be expressed in \DLRp with the axioms:
%
\begin{center}
$R\equiv R_1\!\stackrel[U^1=U^2]{}{\mathlarger{\mathlarger{\mathlarger{\bowtie}}}}\cdots\stackrel[U^{h-1}=U^{h}]{}{\mathlarger{\mathlarger{\mathlarger{\bowtie}}}}\!R_h$

$R: U^1_{1},\ldots, U^1_{n_1},\ldots,U^h_{1},\ldots, U^h_{n_h} \rightarrow U^1$
\end{center}
%
\noindent
where $\tau(R_i)=\{U^i,U^i_1,\ldots, U^i_{n_i}\}$ with $1\leq i\leq h$, and $R$ is a fresh new relation name with signature
%
$\tau(R)=\{U^1,U^1_{1},\ldots, U^1_{n_1},\ldots,U^h_{1},\ldots, U^h_{n_h}\}$.

\emph{Identification} axioms as defined in \DLRID~\cite{CalvaneseGL01}
are a variant of external uniqueness axioms, constraining only the elements of a concept C; they can be expressed in \DLRp with the following axiom:
%
\begin{align*}
& [U^1]\selects{U_1}{C}{R_1}\downarrow\ldots \downarrow [U^h]\selects{U_h}{C}{R_h}.
\end{align*}
%
\noindent
The \DLRID description logic introduced by~\cite{CalvaneseGL01} extends \DLR with functional dependencies and identification axioms, and therefore it is included in \DLRp.

The $\mathcal{CFD}$ family of feature-based description logics have been designed primarily to support efficient PTIME reasoning services about object relational data sources~\cite{TomanW09}. $\mathcal{CFD}$ includes \emph{path functional dependencies} which can be expressed in \DLRp as identification axioms involving joined sequences of functional binary relations.

Since a \DLRp TBox can express both inclusion axioms among concepts and among arbitrary projections of relations, and arbitrary functional dependency axioms, the entailment problem is undecidable~\cite{ChandraV85}.

%%%%%%%%%%%%%%%%%%%%%%%%%%%%%%%%%%%%%%%%%%%%%%%%%%%%%%%%%%%%%%%%%%%%%%
%%%%%%%%%%%%%%%%%%%%%%%%%%%%%%%%%%%%%%%%%%%%%%%%%%%%%%%%%%%%%%%%%%%%%%

\section{The \DLRpm fragment of \DLRp}

Given a \DLRp knowledge base $(\Tmc,\Re)$, the \emph{projection signature} is the set $\mathscr{T}$ 
containing the signatures $\tau(R\!N)$ of the relations $R\!N\in\mathcal{R}$, the singletons associated with each attribute 
name $U\in\mathcal{U}$, and the relation signatures that appear explicitly in projection constructs in some relation 
inclusion axiom from \Tmc, together with their implicit occurrences due to the renaming schema. Formally, 
$\mathscr{T}$ is the smallest set such that
%
\begin{enumerate}
	\item $\tau(R\!N)\in\mathscr{T}$ for all $R\!N\in\mathcal{R}$; 
	\item $\{U\}\in\mathscr{T}$ for all $U\in\mathcal{U}$; and
	\item $\{U_1,\ldots,U_k\}\in\mathscr{T}$ for all $\EXISTR{q}{V_1,\ldots,V_k} R$ appearing in $\Tmc$ and 
		$\{U_i,V_i\}\subseteq [U_i]_\Re $ \text{ for}~$1\!\leq\!i\!\leq\!k$. 
\end{enumerate}
%
The \emph{projection signature graph} is the directed acyclic graph $(\supset,\mathscr{T})$ whose sinks are the attribute 
singletons $\{U\}$. 
%The \DLRpm fragment of \DLRp allows only for knowledge bases whose projection 
%signature graph is a \emph{multitree}; i.e., the set of nodes reachable from any node of the projection signature graph 
%form a tree.  
%
Given a set of attributes $\tau=\{U_1,\ldots,U_k\}\subseteq\mathcal{U}$, the
the \emph{projection signature graph dominated by $\tau$}, denoted as $\mathscr{T}_\tau$, is the subgraph 
of $(\supset,\mathscr{T})$ containing all the nodes reachable from $\tau$.
%
%We call $\mathscr{T}_{\{U_1,\ldots,U_k\}}$ the tree formed by the nodes in the projection signature graph dominated by the set of attributes $\{U_1,\ldots,U_k\}$.
%
Given two sets of attributes $\tau_1,\tau_2\subseteq\mathcal{U}$, $\pth{\tau_1}{\tau_2}$ denotes the set of paths in 
$(\supset,\mathscr{T})$ between $\tau_1$ and $\tau_2$. Note that $\pth{\tau_1}{\tau_2}=\emptyset$ both, when a path 
does not exist and when $\tau_1\subseteq \tau_2$.
The notation $\chd{\tau_1}{\tau_2}$ means that ${\tau_2}$ is a child of ${\tau_1}$ in $(\supset,\mathscr{T})$.
%
\begin{definition}
A \emph{\DLRpm knowledge base} is a \DLRp KB that satisfies the following two conditions:
\begin{enumerate}
\item the projection signature graph $(\supset,\mathscr{T})$ forms a multitree; i.e., for every node
	$\tau\in\mathscr{T}$, the graph $\mathscr{T}_\tau$ is a tree; and
\item for every projection construct $\EXISTR{q}{U_1,\ldots,U_k} R$ appearing in \Tmc, if $q>1$ then the
	length of the path $\pth{\tau(R)}{\{U_1,\ldots,U_k\}}$ is 1.
\end{enumerate}
\end{definition}
%
Essentially, the conditions in \DLRpm restrict \DLRp in the way that multiple projections of relations may appear
in the knowledge base. 
%The multitree condition of \DLRpm restricts \DLRp in the way multiple projections of relations appear in the knowledge base. 
In particular, observe that in \DLRpm $\textsc{path}_{\mathscr{T}}$ is necessarily functional, due to the 
multitree restriction.
Figure~\ref{fig:multitree} shows that the projection signature graph of the knowledge base from Example~\ref{exa:basic}
is indeed a multitree.
%
\begin{figure*}[t]
\begin{center}
\tikz[x=6em,y=8ex] {
\node (a) at (2,3)  {$\left\{\substack{U_1\\V_1\\\ww{P}},\substack{U_2\\V_2\\\ww{P}},\substack{U_3\\V_3\\W_1},\substack{U_4\\V_4\\W_2},\substack{U_5\\V_5\\W_3}\right\}$};
\node (b) at (4,3)  {$\left\{\substack{U_3\\V_3\\W_1},\substack{U_4\\V_4\\W_2},\substack{U_5\\V_5\\W_3},\substack{\ww{P}\\\ww{P}\\W_4}\right\}$};
\node (c) at (1,2)  {$\left\{\substack{U_1\\V_1},\substack{U_2\\V_2}\right\}$};
\node (c1) at (4,1)  {$\left\{\substack{U_5\\V_5\\W_3}\right\}$};
\node (d) at (3,2)  {$\left\{\substack{U_3\\V_3\\W_1},\substack{U_4\\V_4\\W_2},\substack{U_5\\V_5\\W_3}\right\}$};
\node (d1) at (5,2)  {$\left\{\substack{~\\W_4}\right\}$};
\node (e) at (0,1)  {$\left\{\substack{U_1\\V_1}\right\}$};
\node (f) at (1,1)  {$\left\{\substack{U_2\\V_2}\right\}$};
\node (g) at (3,1)  {$\left\{\substack{U_3\\V_3\\W_1},\substack{U_4\\V_4\\W_2}\right\}$};
\node (h) at (2,0)  {$\left\{\substack{U_3\\V_3\\W_1}\right\}$};
\node (i) at (4,0)  {$\left\{\substack{U_4\\V_4\\W_2}\right\}$};
\draw 
(a) edge[->] (c)
(d) edge[->] (c1)
(a) edge[->] (d)
(b) edge[->] (d)
(b) edge[->] (d1)
(c) edge[->] (e)
(c) edge[->] (f) 
(d) edge[->] (g)
(g) edge[->] (h) 
(g) edge[->] (i);
}
\end{center}
\caption{\label{fig:multitree} The projection signature graph of the example.}
\end{figure*}
%
Note that in the figure we have collapsed equivalent attributes in a unique equivalence class, according to the renaming 
schema. Since all its projection constructs $q=1$, this knowledge base belongs to \DLRpm.

%In addition to the above multitree condition, the \DLRpm fragment of \DLRp allows for knowledge bases with projection constructs $\EXISTR{q}{U_1,\ldots,U_k} R$ (resp. $\EXISTR{q}{U} R$) with a cardinality $q>1$ only if the length of the path $\pth{\{U_1,\ldots,U_k\}}{\tau(R)}$ (resp. $\pth{\{U\}}{\tau(R)}$) is 1.

It is easy to see that \DLR is included in \DLRpm, since the projection signature graph of any \DLR knowledge base is always a degenerate multitree with maximum depth equal to 1. 

\DLRID~\cite{CalvaneseGL01} extended with unary functional dependencies is also included in \DLRpm, with the proviso that projections of relations in the knowledge base form a multitree projection signature graph. Since (unary) functional dependencies are expressed via the inclusions of projections of relations (see, e.g., the functional dependency~(\ref{funct-dep}) in the previous example), by constraining the projection signature graph to be a multitree, the possibility to build combinations of functional dependencies as the ones in~\cite{CalvaneseGL01} leading to undecidability is ruled out. 

Also note that \DLRpm is able to correctly express the UML fragment as introduced in~\cite{BeCD05-AIJ-2005,ACKRZ:er07} and the ORM fragment as introduced in~\cite{DBLP:conf/otm/FranconiM13}.

%%%%%%%%%%%%%%%%%%%%%%%%%%%%%%%%%%%%%%%%%%%%%%%%%%%%%%%%%%%%%%%%%%%%%%
%%%%%%%%%%%%%%%%%%%%%%%%%%%%%%%%%%%%%%%%%%%%%%%%%%%%%%%%%%%%%%%%%%%%%%

\section{Mapping \DLRpm to \ALCQI}
\label{sec:mapping}

We show that reasoning in \DLRpm is \ExpTime-complete by providing a
mapping from \DLRpm knowledge bases to \ALCQI knowledge bases; the
reverse mapping from \ALCQI knowledge bases to \DLR knowledge bases is
well known. The proof is based on the fact that reasoning with \ALCQI
knowledge bases is \ExpTime-complete~\cite{BCMNP03}. We adapt and
extend the mapping presented for \DLR in~\cite{calvanese:et:al:98b}
and\nb{A: added} then adapted by~\cite{HorrocksSTT00} to deal with
ABoxes possibly without the UNA.

%We quickly remind that \ALCQI \emph{concepts} $C$ and \emph{roles} $S$ are defined as follows:
%%
%\begin{align*}
%C  &\to \top\ \mid\ \bot\  \mid\ A_i ~\mid~  \neg C ~\mid~ C_1 \sqcap C_2
%~\mid~ C_1 \sqcup C_2 ~\mid~ \card{q}{S_i}{C} ~\mid~
%\forall S_i\per C,\\
%S &\to Q_i\ \mid Q_i^-,
%\end{align*}
%%
%where $A_i$ is a concept name and $Q_i$ a role name. The non-boolean constructors are interpreted in an interpretation $\mathcal{I} = (\Delta,\Int \cdot)$ by taking
%%
%\begin{align*}
%  (\card{q}{S_i}{C})^\mathcal{I} &= \{ x\in\Delta \mid~\left|\{y\in
%                  \Delta \mid
%                  (x,y)\in S_i^\mathcal{I} \wedge y\in C^\mathcal{I}\}\right| \lesseqgtr q \},\\
%%
%  (\forall S_i\per C)^\mathcal{I} &= \{ x\in\Delta \mid \forall y\in \Delta\
%                                 ((x,y)\in S_i^\mathcal{I} \to y\in C^\mathcal{I}) \},\\
%%
%  \Int{(Q_i^-)} &= \{(x,y)\in \Delta\times \Delta \mid (y,x)\in \Int
%Q\}.
%\end{align*}

In the following we use the shortcut $(S_1\chain\ldots\chain S_n)^-$
for $S_n^-\chain\ldots\chain S_1^-$, the shortcut
$\exists^{\lesseqgtr 1} S_1\chain\ldots\chain S_n\per C$ for
$\card{1}{S_1\per\ldots\per\exists^{\lesseqgtr 1} S_n}{C}$, the
shortcut $\forall S_1\chain\ldots\chain S_n\per C$ for
$\forall S_1\per\ldots\per\forall S_n\per {C}$, while\nb{A: changed}
the concept expressions $\exists^{\lesseqgtr 1}\bot\per{C}$ and
$\forall \bot\per {C}$ have to be considered  as the $\bot$ concept.
Note\nb{A: sentence changed, the original is commented} that the
shortcut for qualified number restrictions is limited to the case
$q = 1$.
%
% for the role chain constructor ``$\chain$'' are not correct
% in general, but they are correct in the context of the specific \ALCQI
% knowledge bases used in this paper.
%
For a relation instance axiom in the ABox of the form
$R\!N(U_1\!:\!o_1,\ldots,U_n\!:\!o_n)$ we use the shortcut $R\!N(t)$,
with $t = \langle U_1\!:\!o_1,\ldots,U_n\!:\!o_n\rangle$ a
\emph{relation instance}, namely a $\mathcal{U}$-labelled tuple over
the set of individuals in $\Ob$.

Let $\KB = (\mathcal{T},\A,\Re)$ be a \DLRpm knowledge base.
We first preprocess the \DLRpm knowledge base by transforming it into a logically equivalent one as follows: for each equivalence class $[U]_{\Re}$ a single \emph{canonical} representative of the class is chosen, and the \KB is consistently rewritten by substituting each attribute with its canonical representative. After this rewriting, the renaming schema does not play any role in the mapping.

%%%%%%%%%%%%%%%%%%%%%%%%%%%

\begin{figure*}
	[t] 
	\begin{center}
		\renewcommand{\arraystretch}{1.2} $ 
		\begin{array}{r@{\hspace{1ex}}c@{\hspace{1ex}}l} 
			(\neg C)^\dag &=& \neg C^\dag \\
			%
			(C_1 \sqcap C_2)^\dag & =& C_1^\dag \sqcap C_2^\dag \\
			%
			(C_1 \sqcup C_2)^\dag & =& C_1^\dag \sqcup C_2^\dag \\
			%
			(\EXISTR{q}{U_i} R)^\dag & =& \card{q}{\left(\pth{\tau(R)}{\{U_i\}}^\dag\right)^-}{R^\dag}\\
			%
			(\greif{R})^\dag &=& R^\dag \\
			%
			(\lreif{R\!N})^\dag &=& A^l_{R\!N}
			%
			%
			\vspace{2ex}\\
			%
			%
			(R_1\setminus R_2)^\dag &=& R_1^\dag\sqcap \neg R_2^\dag\\
			%
			(R_1 \sqcap R_2)^\dag & =& R_1^\dag \sqcap R_2^\dag\\
			%
			(R_1 \sqcup R_2)^\dag & =& R_1^\dag \sqcup R_2^\dag\\
			%
			(\selects{U_i}{C}{R})^\dag & =& R^\dag \sqcap \forall \pth{\tau(R)}{\{U_i\}}^\dag \per C^\dag\\
			%
			(\EXISTR{q}{U_1,\ldots,U_k} R)^\dag & =&
			%  \begin{cases}
			%    {R^\dag} & \parbox[t]{\textwidth}{$\text{if } q=1 \text{ and }\\ \{U_1,\ldots,U_k\}\!=\!\tau(R)$}\\
			\card{q}{\left(\pth{\tau(R)}{\{U_1,\ldots,U_k\}}^\dag\right)^-}{R^\dag} 
			%    & \text{otherwise}
			% \end{cases}                                   
		\end{array}
		$ 
		\renewcommand{\arraystretch}{1} 
	\end{center}
	\caption{\label{fig:themapping} The mapping for concept and relation expressions.} 
\end{figure*}

Let's first introduce a mapping function $\cdot^\dag$ from \DLRpm
concept and relation expressions to \ALCQI concepts. Starting with
atomic expressions, the mapping function $\cdot^\dag$ maps each
concept name $C\!N$ in the \DLRpm knowledge base to an \ALCQI concept
name $C\!N$ and each relation name $R\!N$ in the \DLRpm knowledge base to
an \ALCQI concept name $A_{R\!N}$ (its global reification).
%
For each relation name $R\!N$, the \ALCQI signature also
includes a concept name $A_{R\!N}^{l}$ and a role name $Q_{R\!N}$, in
order to capture the local objectification. The mapping $\cdot^\dag$
is extended to concept and relation expressions as in
Figure~\ref{fig:themapping}.

The mapping crucially uses the projection signature graph structure to map projections and selections, by accessing paths in the projection signature $\mathscr{T}$ associated to the \DLRpm knowledge base. If there is a path from ${\tau}$ to ${\tau'}$ in $\mathscr{T}$---i.e., $\pth{\tau}{\tau'} = \tau,\tau_1,\ldots,\tau_n, \tau'$---then its mapping is an \ALCQI  role chain expression using role names $Q_{\tau_i}$ as follows:
%
\begin{align*}
	\pth{\tau}{\tau'}^\dag = Q_{\tau_1}\chain\ldots\chain Q_{\tau_n}\chain Q_{\tau'}. 
\end{align*}
%
\noindent with $\pth{\tau}{\tau'}^\dag = \bot$ if $\pth{\tau}{\tau'} = \emptyset$.

%Furthermore, the \ALCQI signature also includes a role name $Q_{\tau_i}$ for each projected signature $\tau_i$ in the projection signature, $\tau_i\in\mathscr{T}$, such that there exists $\tau_j\in\mathscr{T}$ with $\chd{\tau_j}{\tau_i}$, i.e., we exclude the case where $\tau_i$ is one of the roots of the multitree induced by the projection signature.

The \ALCQI signature also includes a concept name $A_{R\!N}^{\tau_i}$ for each projected signature $\tau_i$ in the projection signature graph dominated by $\tau(R\!N)$, $\tau_i\in\mathscr{T}_{\tau(R\!N)}$ (to capture global reifications of the projections of $R\!N$). Note that $A_{R\!N}^{\tau(R\!N)}$ coincides with $A_{R\!N}$. 

Intuitively, the mapping reifies each node in the projection signature graph: the target \ALCQI signature of the example of the previous section is partially presented in Fig.~\ref{fig:mapping}, together with the projection signature graph. Each node is labelled with the corresponding (global) reification concept ($A_{R_i}^{\tau_j}$), for each relation name $R_i$ and each projected signature $\tau_j$ in the projection signature graph dominated by $\tau(R_i)$, while the edges are labelled by the roles ($Q_{\tau_i}$) needed for the reification.

\begin{figure*}[t]
\begin{center}
\tikz[x=6em,y=10ex] {\tiny
\node (a) at (2,3)  {$A_{R_1}, A_{R_2}$};
\node (b) at (4,3)  {$A_{R_3}$};
\node (c) at (1,2)  {$A_{R_1}^{\{U_1,U_2\}}, A_{R_2}^{\{U_1,U_2\}}$};
\node (c1) at (5,1)  {$A_{R_1}^{\{U_5\}}, A_{R_2}^{\{U_5\}}, A_{R_3}^{\{U_5\}}$};
\node (d) at (3,2)  {$A_{R_1}^{\{U_3,U_4,U_5\}}, A_{R_2}^{\{U_3,U_4,U_5\}}, A_{R_3}^{\{U_3,U_4,U_5\}}$};
\node (d1) at (5,2)  {$A_{R_3}^{\{W_4\}}$};
\node (e) at (0,1)  {$A_{R_1}^{\{U_1\}}, A_{R_2}^{\{U_1\}}$};
\node (f) at (1,1)  {$A_{R_1}^{\{U_2\}}, A_{R_2}^{\{U_2\}}$};
\node (g) at (3,1)  {$A_{R_1}^{\{U_3,U_4\}}, A_{R_2}^{\{U_3,U_4\}}, A_{R_3}^{\{U_3,U_4\}}$};
\node (h) at (2,0)  {$A_{R_1}^{\{U_3\}}, A_{R_2}^{\{U_3\}}, A_{R_3}^{\{U_3\}}$};
\node (i) at (4,0)  {$A_{R_1}^{\{U_4\}}, A_{R_2}^{\{U_4\}}, A_{R_3}^{\{U_4\}}$};
\node (ac) at (1.3,2.7) {$Q_{{\{U_1,U_2\}}}$};
\node (ad) at (2.75,2.7) {$Q_{\{U_3,U_4,U_5\}}$};
\node (bd) at (4,2.5) {$Q_{\{U_3,U_4,U_5\}}$};
\node (bd1) at (4.75,2.5) {$Q_{\{W_4\}}$};
\node (ce) at (0.3,1.6) {$Q_{{\{U_1\}}}$};
\node (cf) at (1.23,1.6) {$Q_{{\{U_2\}}}$};
\node (dg) at (2.7,1.6) {$Q_{{\{U_3,U_4\}}}$};
\node (dc1) at (4.2,1.6) {$Q_{{\{U_5\}}}$};
\node (gh) at (2.2,0.5) {$Q_{\{U_3\}}$};
\node (gi) at (3.75,0.5) {$Q_{\{U_4\}}$};
\draw 
(a) edge[->] (c)
(a) edge[->] (d)
(d) edge[->] (c1)
(b) edge[->] (d)
(b) edge[->] (d1) 
(c) edge[->] (e)
(c) edge[->] (f) 
(d) edge[->] (g)
(g) edge[->] (h) 
(g) edge[->] (i);
}
\end{center}
\caption{\label{fig:mapping} The \ALCQI signature generated by the example.}
\end{figure*} 

%%%%%%%%%%%%%%%%%%%%%%%%%%%

Note that in \DLRpm the cardinalities on a path are restricted to the
case $q=1$ whenever a path is of length greater than $1$, so we still
remain within the \ALCQI description logic when the mapping applies to
cardinalities. So, if we need to express a cardinality constraint
$\EXISTR{q}{U_i} R$ (or $\EXISTR{q}{U_1,\ldots,U_k} R$), with $q>1$,
then $U_i$ ($\{U_1,\ldots,U_k\}$) should not be mentioned in any other
projection of the relation $R$ in such a way that
$|\pth{\tau(R)}{\{U_i\}}|=1$ ($|\pth{\tau(R)}{\{U_1,\ldots,U_k\}}|=1$).

In order to explain the need for the path function in the mapping,
notice that a relation is reified according to the decomposition
dictated by the projection signature graph it dominates. Thus, to
access an attribute $U_j$ of a relation ${R_i}$ it is necessary to
follow the path through the projections that use that attribute. This
path is a role chain from the signature of the relation (the root) to
the attribute as returned by the $\pth{\tau(R_i)}{U_i}$ function. For
example, considering Fig.~\ref{fig:mapping}, in order to access the
attribute $U_4$ of the relation $R_3$ in the expression
$(\selects{U_4}{C}{R_3})$, the path $\pth{\tau(R_3)}{\{U_4\}}^\dag$ is
equal to the role chain
$Q_{\{U_3,U_4,U_5\}}\chain Q_{\{U_3,U_4\}}\chain Q_{\{U_4\}}$, so that
$ (\selects{U_4}{C}{R_3})^\dag ~=~ A_{R_3} \sqcap \forall
Q_{\{U_3,U_4,U_5\}}\chain Q_{\{U_3,U_4\}}\chain Q_{\{U_4\}}\per C.  $
\\
Similar considerations can be done when mapping cardinalities over
relation projections.

Let $\KB = (\Tmc, \A)$ be a \DLRpm knowledge base with a signature
$(\mathcal{C},\mathcal{R},\mathcal{U},\tau)$. The mapping
$\gamma(\KB) = (\gamma(\Tmc),\gamma(\A))$ is defined as the following
\ALCQI KB:
%
\begin{align*}
  \gamma(\KB) \quad =& \quad \gamma_{\textit{dsj}} ~\cup %
  \bigcup_{R\!N\in\Rmc}\gamma_{\textit{rel}}(R\!N) ~\cup %
  \bigcup_{R\!N\in\Rmc}\gamma_{\textit{lobj}}({R\!N})
  ~\cup \\
%  \bigcup_{{R\!N}\in\Rmc, P_i\in \mathscr{T}_{R\!N}}\hspace*{-1.7em}\gamma_{\textit{part}}({P_i})\quad\cup\quad %
% \bigcup_{U_1\looparrowright U_2\in\KB}\hspace*{-1em}{U_1 \equiv U_2}\quad\cup\quad %
  & \quad\bigcup_{C_1\sqsubseteq C_2\in\KB}{C_1^\dag\sqsubseteq C_2^\dag}
  ~\cup%
 \bigcup_{R_1\sqsubseteq R_2\in\KB}{R_1^\dag\sqsubseteq
    R_2^\dag},%
\end{align*}
%
where
%
\begin{align*}
\gamma_\textit{dsj} = % &~\bigl\{A\sqsubseteq\neg A_{R\!N}
  % \mid A\in\Cmc ~\mbox{and}~ {R\!N}\in\Rmc\bigr\} ~\cup \\
&~\bigl\{A_{R\!N_1}^{\tau_i}\sqsubseteq\neg A_{R\!N_2}^{\tau_j} \mid R\!N_1, R\!N_2\in\Rmc,
  \tau_i, \tau_j\in \mathscr{T}, |\tau_i|\geq 2, |\tau_j|\geq 2, \tau_i\neq \tau_j
  \bigr\}\\
% &\bigl\{{R\!N_1^\dag}\sqsubseteq\neg {R\!N_2^\dag} \mid R\!N_1, R\!N_2\in\Rmc,
%   \exists P_i\in \mathscr{T}_{R\!N_1}\per \forall P_j\in \mathscr{T}_{R\!N_2}\per [P_i]_{\Re} \neq [P_j]_{\Re}
%   \bigr\};\\
%note that $A_2^\dag$ includes also all the concepts names of the form $A_{R\!N}^{P_i}$;
%
  \gamma_{\textit{rel}}(R\!N) =&~\bigcup_{\tau_i\in\mathscr{T}_{\tau(R\!N)}}~
     \bigcup_{\chd{\tau_i}{\tau_j}}\bigl\{
     A^{\tau_i}_{R\!N} \sqsubseteq \exists Q_{\tau_j}\per A^{\tau_j}_{R\!N},~\exists^{\geq 2}
     Q_{\tau_j}\per\top\sqsubseteq \bot
     \bigr\} \\
% ~\cup\\
% &\bigcup_{\textit{child}(\tau_i,U_j)}\hspace*{-1.5em}\bigl\{
%    A^{P_i}_{R\!N} \sqsubseteq \exists {U_j}\per\top,~\exists^{\geq 2}
%    {U_j}\per\top\sqsubseteq \bot\bigr\}\Bigr);\\
%
\gamma_{\textit{lobj}}({R\!N}) =&~\{\parbox[t]{\textwidth}{$
A_{R\!N}\sqsubseteq\exists Q_{R\!N}\per A_{R\!N}^l,~
\exists^{\geq 2}Q_{R\!N}\per \top\sqsubseteq \bot,\\
A_{R\!N}^l \sqsubseteq \exists Q_{R\!N}^-\per A_{R\!N},~
\exists^{\geq 2} Q_{R\!N}^-\per \top\sqsubseteq \bot\}$}
\end{align*}
%
Intuitively, $\gamma_\textit{dsj}$ ensures that relations with
different signatures are disjoint, thus, e.g., enforcing the union
compatibility. The axioms in $\gamma_{\textit{rel}}$ introduce
classical reification axioms for each relation and its relevant
projections. The axioms in $\gamma_{\textit{lobj}}$ make sure that
each local objectification differs form the global one. 

As for the mapping of the ABox, we map each individual $O$ in the
\DLRpm ABox to an \ALCQI individual $O$ (the mapping for
individuals behaves as the identity function).
%
Each relation instance occurring in \A (we say that a relation
instance $t$ \emph{occurs} in \A if $R\!N(t)\in\A$ for some relation
name $R\!N$) is mapped via an injective function, $\xi$, to a distinct
individual, i.e., $\xi: T_\Ob(\mathcal{U})\to \mathcal{O}_\ALCQI$,
with $\mathcal{O}_\ALCQI = \Ob\cup \Ob^t$ being the set of individuals
in $\gamma(\KB)$, $\Ob \cap \Ob^t = \emptyset$ and

$\xi(t)  ~=~ %\left\{
                  \begin{cases}
                    o\in\Ob, &\text{if } t = \langle U\!:\!o\rangle \\
                    o\in\Ob^t, &\text{otherwise.}\\
                  \end{cases}
$

\noindent The mapping $\gamma(\A)$ of the ABox, similarly to the
mapping presented in~\cite{HorrocksSTT00}, introduces a new concept
name $Q_o$ for every individual in $o \in\Ob$ and a new concept name
$Q_t$ for every relation instance occurring in \A. Then, $\gamma(\A)$
is as follows:
%
\begin{align}
\gamma(\A) =
& \label{eq:ob1}
\{ C\!N^\dag(o) \mid C\!N(o)\in\A \} ~\cup\\
& \label{eq:ob2}
\{ o_1\neq o_2 \mid o_1\neq o_2\in\A \} ~\cup~\{ o_1=o_2 \mid o_1= o_2\in\A \}~\cup\\
& \label{eq:rei1}
\{A^{\tau_i}_{R\!N}(\xi(t[\tau_i])) \mid R\!N(t)\in\A \textit{ and } \tau_i
  \in\mathscr{T}_{\tau(R\!N)}\} ~\cup\\
& \label{eq:rei2}
\{ Q_{\tau_j}\big(\xi(t[\tau_i]),\xi(t[\tau_j])\big) \mid
  \chd{\tau_i}{\tau_j} \}\\
& \label{eq:unique1}
\{Q_o(o)\mid o\in\Ob\}~\cup\\
& \label{eq:unique2}
\{Q_t(o_1),~Ax(Q_t)\mid t = \langle
  U_1\!:\!o_1,\ldots, U_n\!:\!o_n\rangle \text{ occurs in } \A\},
\end{align}
%
where $Ax(Q_t)$ stands for the following axiom:
%
\begin{multline}
  \label{eq:unique3}
  Q_t\sqsubseteq \exists^{\leq 1}\big(\pth{\tau(t)}{\{U_1\}}^\dag
  \big)^-\per\Big(\exists \big(\pth{\tau(t)}{\{U_2\}}^\dag
  \big)\per Q_{o_2}\sqcap\ldots\sqcap\\
  \exists \big(\pth{\tau(t)}{\{U_n\}}^\dag
  \big)\per Q_{o_n}\Big)
\end{multline}
%
Intuitively, \eqref{eq:rei1} and \eqref{eq:rei2} reify each relation
instance occurring in \A using the projection signature of the
relation instance itself. The formulas \eqref{eq:unique1}-\eqref{eq:unique3} guarantee that there is exactly one \ALCQI
individual reifying a given relation instance.

Clearly, the size of $\gamma(\KB)$ is polynomial in the size of $\KB$
(under the same coding of the numerical parameters), and thus we are
able to state the main result of this paper.\nb{A: the Lemma must be
  moved to another Section!}

\begin{lemma}
	The problems of concept and relation satisfiability and of entailment in \DLRpm are reducible to \DLRpm KB satisfiability.
\end{lemma}

\begin{proof}
TO BE PROVED
%	We first show that we can reduce all the problems to concept satisfiability, where a concept $C$ is satisfiable iff $\KB \nvDash C\sqsubseteq \bot$. 
%%
%	\begin{itemize}
%		\item \KB is satisfiable iff $\KB \nvDash \top\sqsubseteq \bot$; 
%		\item $\KB \models C_1 \sqsubseteq C_2$ iff $\KB \models C_1 \sqcap \neg C_2\sqsubseteq \bot$; 
%		\item $\KB \models R_1 \sqsubseteq R_2$ iff $\KB \models \exists[U](R_1\sqcap \neg R_2)\sqsubseteq \bot$, for some $U\in\tau(R_1)$; 
%		\item $\KB \nvDash R \sqsubseteq \bot$ iff $\KB \nvDash \exists[U]R\sqsubseteq \bot$, for some $U\in\tau(R)$. 
%	\end{itemize}
%%
%	Viceversa, we can show that concept satisfiability can be reduced to any other problem. First, note that concept satisfiability is already expressed as a logical implication problem. For the other cases, given a fresh new binary relation $P$, we have that 
%%
%	\begin{itemize}
%		\item $\KB \nvDash C\sqsubseteq \bot$ iff $\KB \cup \{\top \sqsubseteq \exists[U_1](P\sqcap \sigma_{U_2:C}P)\}$ is satisfiable; 
%		\item $\KB \nvDash C\sqsubseteq \bot$ iff $\KB\nvDash \sigma_{U_2:C}P \sqsubseteq \bot$.\hfill\qed 
%	\end{itemize}
%	
\end{proof}

\begin{theorem}
	\label{th:sat} A \DLRpm knowledge base $\KB$ is satisfiable iff the \ALCQI knowledge base $\gamma(\KB)$ is satisfiable. 
\end{theorem}
%
\begin{proof}
  We assume that the \KB is consistently rewritten by substituting
  each attribute with its canonical representative, thus, we do not
  have to deal with the renaming of attributes. Furthermore, we extend
  the function $\imath$ to
  singleton tuples with the meaning that $\imath(\langle
  U_i:d_i\rangle)=d_i$.\\
%
  ($\Rightarrow$) Let
  $\Imc = (\Delta^\Imc, \cdot^\Imc, \rho, \imath,
  \ell_{R\!N_1},\ldots)$ be a model for a \DLRpm knowledge base
  $\KB$. To construct a model $\Jmc=(\Delta^\Jmc,\cdot^\Jmc)$ for the
  \ALCQI knowledge base $\gamma(\KB)$ we set
  $\Delta^\Jmc = \Delta^\Imc$, $o^\Jmc = o^\Imc$ for all $o\in\Ob$ and
%
  \begin{align}
    \label{int:xi}    
    [\xi(\langle U_1\!:\!o_1,\ldots,U_n\!:\!o_n\rangle)]^\Jmc =
    [\imath(\langle
    U_1\!:\!o^\Imc_1,\ldots,U_n\!:\!o^\Imc_n\rangle)].
  \end{align}
%
  Furthermore, we set:
  $(C\!N^\dag)^\Jmc=(C\!N)^\Imc$, for every atomic concept
  $C\!N\in\Cmc$, while for every ${R\!N}\in\Rmc$ and
  ${\tau_i\in\mathscr{T}_{\tau(R\!N)}}$ we set
  % $(A_{R\!N}^{\tau_i})^\Jmc = \Gamma_{R\!N}^{\tau_i}$, 
  % where
  % $\Gamma_{R\!N} = \bigcup_{\tau_i\in\mathscr{T}_{\tau(R\!N)}}
  % \Gamma_{R\!N}^{\tau_i}$ and
  \begin{multline}\label{eq:R}
    (A_{R\!N}^{\tau_i})^\Jmc = \{\imath(\langle U_1:d_1,\ldots,U_k:d_k\rangle) \mid 
    \{U_1,\ldots,U_k\}=\tau_i \text{ and } \\ \exists t\in R\!N^\Imc\per t[U_1]=d_1,\ldots,t[U_k]=d_k\}.
  \end{multline}
%
  For each role name $Q_{\tau_i}$, $\tau_i\in\mathscr{T}$, we set
  \begin{multline}\label{eq:Q}
    (Q_{\tau_i})^\Jmc = \{(d_1,d_2)\in \Delta^\Jmc\times
    \Delta^\Jmc\mid \exists t\in {R\!N}^\Imc \text{ s.t. } d_1=\imath(t[{\tau_j}]), d_2=\imath(t[{\tau_i}])\\
    \text{ and } \chd{\tau_j}{\tau_i}, \text{ for some } {R\!N}\in\R\}.
  \end{multline}
%
  For every ${R\!N}\in\Rmc$ we set
  \begin{multline}\label{eq:loc}
  Q_{R\!N}^\Jmc = \{(d_1,d_2)\in \Delta^\Jmc\times \Delta^\Jmc\mid
    \exists t\in\Int{R\!N} \text{ s.t. } d_1=\imath(t) \text{ and } d_2=\ell_{R\!N}(t)\},
  \end{multline}
%
and
%
  \begin{align}\label{eq:Aloc}
  (A^l_{R\!N})^\Jmc = \{\ell_{R\!N}(t)\mid t\in \Int{R\!N}\}.
  \end{align}
%
We first show that $\Jmc$ is indeed a model of $\gamma(\Tmc)$.
\begin{enumerate}
\item $\Jmc\models \gamma_\textit{dsj}$. This is a direct consequence
  of the fact that $\imath$ is an injective function and that tuples
  with different aryties are different tuples.
%
\item $\Jmc\models \gamma_{\textit{rel}}(R\!N)$, for every
  ${R\!N\in\Rmc}$. We show that, for each $\tau_i,\tau_j$
  s.t. $\chd{\tau_i}{\tau_j}$ and $\tau_i\in\mathscr{T}_{\tau(R\!N)}$,
  $\Jmc\models A^{\tau_i}_{R\!N} \sqsubseteq \exists Q_{\tau_j}\per
  A^{\tau_j}_{R\!N}$
  % $\Jmc\models A^{\tau_i}_{R\!N} \sqsubseteq \forall Q_{\tau_j}\per A^{\tau_j}_{R\!N}$ 
  and
  $\Jmc\models~\exists^{\geq 2} Q_{\tau_j}\per\top\sqsubseteq \bot$:
  \begin{itemize}
  \item
    $\Jmc\models A^{\tau_i}_{R\!N} \sqsubseteq \exists Q_{\tau_j}\per
    A^{\tau_j}_{R\!N}$.
    Let $d\in(A^{\tau_i}_{R\!N})^\Jmc$, by~(\ref{eq:R}),
    $\exists t\in {R\!N}^\Imc$ s.t. $d=\imath(t[{\tau_i}])$. Since
    $\chd{\tau_i}{\tau_j}$, then $\exists d'=\imath(t[{\tau_j}])$ and,
    by~(\ref{eq:Q}), $(d,d')\in Q_{\tau_j}^\Jmc$, while
    by~(\ref{eq:R}), $d'\in (A^{\tau_j}_{R\!N})^\Jmc$. Thus,
    $d\in (\exists Q_{\tau_j}\per A^{\tau_j}_{R\!N})^\Jmc$.
  \item
    $\Jmc\models~\exists^{\geq 2} Q_{\tau_j}\per\top\sqsubseteq \bot$.
    The fact that each $Q_{\tau_j}$ is interpreted as a funcional role is
    a direct consequence of the construction~(\ref{eq:Q}) and the fact
    that $\imath$ is an injective function.
  \end{itemize}
%
\item $\Jmc\models \gamma_{\textit{lobj}}(R\!N)$, for every
  ${R\!N\in\Rmc}$. Similar as above, considering the fact that each $\ell_{R\!N}$ is
  an injective function and equations~(\ref{eq:loc})-(\ref{eq:Aloc}).
%
\item $\Jmc\models {C_1^\dag\sqsubseteq C_2^\dag}$ and
  $\Jmc\models {R_1^\dag\sqsubseteq R_2^\dag}$. Since
  $\Imc\models {C_1\sqsubseteq C_2}$ and
  $\Imc\models R_1\sqsubseteq R_2$, it is enough to show the
  following:
  \begin{itemize}
  \item $d\in \Int{C} \text{ iff } d\in (C^\dag)^\Jmc$, for all \DLRpm\ concepts;
  \item $t\in \Int{R} \text{ iff } \imath(t)\in (R^\dag)^\Jmc$, for all \DLRpm\ relations.
  \end{itemize}
%
  Before we proceed with the proof, it is easy to show by structural
  induction that the following property holds:
  \begin{align}\label{prop:RN1}
    \text{If } \imath(t)\in R^{\dag\Jmc} \text{ then } \exists
    \imath(t')\in R\!N^{\dag\Jmc} 
    \text{ s.t. } t=t'[\tau(R)], \text{ for some } R\!N\in\R.
  \end{align}
%
  We now proceed with the proof by structural induction. The base
  cases, for atomic concepts and roles, are immediate form the
  definition of both ${C\!N}^\Jmc$ and ${R\!N}^\Jmc$.
%
  The cases where complex concepts and relations are constructed using
  either boolean operators or global reification are easy to show. We
  thus show only the following cases.\\
%
  Let $d\in(\lreif R\!N)^\Imc$. Then, $d=\ell_{R\!N}(t)$ with
  $t\in R\!N^\Imc$. By induction, $\imath(t)\in A_{R\!N}^\Jmc$ and, by
  $\gamma_{\textit{lobj}}({R\!N})$, there is a $d'\in\Delta^\Jmc$
  s.t. $(\imath(t),d')\in Q_{R\!N}^\Jmc$ and
  $d'\in (A_{R\!N}^l)^\Jmc$. By~(\ref{eq:loc}), $d'=\ell_{R\!N}(t)$
  and, since $\ell_{R\!N}$ is injective, $d'=d$. Thus,  $d\in(\lreif
  R\!N)^{\dag\Jmc}$.\\
%
  Let $d\in(\exists^{\geq q}[U_i] R)^\Imc$. Then, there are different
  $t_1,\ldots,t_q\in R^\Imc$ s.t. $t_l[U_i]=d$, for all
  $l=1,\ldots,q$. By induction, $\imath(t_l)\in R^{\dag \Jmc}$ while,
  by~(\ref{prop:RN1}), $\imath(t'_l)\in R\!N^{\dag\Jmc}$, for
  some atomic relation ${R\!N}\in\R$ and a tuple $t'_l$
  s.t. $t_l=t'_l[\tau(R)]$. By~$\gamma_{\textit{rel}}({R\!N})$
  and~(\ref{eq:Q}), $(\imath(t'_l),\imath(t_l))\in
  (\pth{\tau({R\!N})}{\tau(R)}^\dag)^\Jmc$ and  $(\imath(t_l),d)\in
  (\pth{\tau({R})}{\{U_i\}}^\dag)^\Jmc$. Since $\imath$ is injective,
  $\imath(t_l)\neq \imath(t_j)$ when $l\neq j$,
  thus, $d\in(\exists^{\geq q}[U_i] R)^{\dag\Jmc}$.\\
%
  Let $t\in(\selects{U_i}{C}{R})^\Imc$. Then, $t\in R^\Imc$ and
  $t[U_i]\in C^\Imc$ and, by induction, $\imath(t)\in R^{\dag \Jmc}$
  and $t[U_i]\in C^{\dag \Jmc}$. As before, by
  $\gamma_{\textit{rel}}(R\!N)$ and by~(\ref{eq:Q}) and (\ref{prop:RN1}),
  $(\imath(t),t[U_i])\in(\pth{\tau(R)}{\{U_i\}}^\dag)^\Jmc$.
   Since $\pth{\tau(R)}{U_i}^\dag$ is functional, then we have that
  $\imath(t)\in (\selects{U_i}{C}{R})^{\dag \Jmc}$.\\
%
  Let $t\in(\exists[U_1,\ldots,U_k] R)^{\Imc}$. Then, there is a tuple
  $t'\in R^\Imc$ s.t. $t'[U_1,\ldots,U_k]=t$ and, by induction,
  $\imath(t')\in R^{\dag\Jmc}$. As before, by
  $\gamma_{\textit{rel}}(R\!N)$ and by~(\ref{eq:Q}) and
  (\ref{prop:RN1}), we can show that
  $(\imath(t'),\imath(t))\in
  \pth{\tau(R)}{\{U_1,\ldots,U_k\}}^{\dag\Jmc}$ and thus
  $\imath(t)\in(\exists[U_1,\ldots,U_k] R)^{\dag\Jmc}$.\\
%
  All the other cases can be proved in a similar way. We now show the
  vice versa.

  \smallskip

  Let $d\in(\lreif R\!N)^{\dag\Jmc}$. Then, $d\in (A_{R\!N}^l)^\Jmc$
  and $d=l_{R\!N}(t)$, for some $t\in
  {R\!N}^\Imc$, i.e., $d\in(\lreif R\!N)^\Imc$.\\
%
  Let $d\in(\exists^{\geq q}[{U_i}] R)^{\dag \Jmc}$. Then, there are
  different  $d_1,\ldots,d_q\in\Delta^\Jmc$ s.t.
  $(d_l,d)\in (\pth{\tau(R)}{\{U_i\}}^\dag)^{\Jmc}$ and
  $d_l\in R^{\dag \Jmc}$, for $l=1,\ldots,q$. By induction, each
  $d_l=\imath(t_l)$ and $t_l\in R^\Imc$. Since $\imath$ is injective,
  then $t_l\neq t_j$ for all $l,j=1,\ldots,q$, $l\neq j$. We need to
  show that $t_l[U_i] = d$, for all $l=1,\ldots,q$. By~(\ref{eq:Q})
  and the fact that $(d_l,d)\in (\pth{\tau(R)}{\{U_i\}}^\dag)^{\Jmc}$,
  then $d=\imath(t_l[U_i])=t_l[U_i]$.\\
%
  Let $\imath(t)\in(\selects{U_i}{C}{R})^{\dag\Jmc}$.  Then,
  $\imath(t)\in R^{\dag \Jmc}$ and, by induction, $t\in R^\Imc$. Let
  $t[U_i]=d$. We need to show that $d\in C^\Imc$. By
  $\gamma_{\textit{rel}}(R\!N)$ and by~(\ref{eq:Q}) and
  (\ref{prop:RN1}),
  $(\imath(t),d)\in (\pth{\tau(R)}{\{U_i\}}^\dag)^\Jmc$, then
  $d\in C ^{\dag\Jmc}$ and, by induction, $d\in C ^{\Imc}$.\\
%  Thus, $t\in(\selects{U_i}{C}{R})^\Imc$.\\
%
  Let $\imath(t)\in(\exists[U_1,\ldots,U_k] R)^{\dag\Jmc}$. Then, there is
  $d\in\Delta^\Jmc$ s.t.
  $$(d,\imath(t)) \in (\pth{\tau(R)}{\{U_1,\ldots,U_k\}}^\dag)^\Jmc$$ and
  $d\in R^\Jmc$.  By induction, $d=\imath(t')$ and
  $t'\in R^\Imc$. By~(\ref{eq:Q}), $\imath(t)=
  \imath(t'[U_1,\ldots,U_k])$, i.e., $t=t'[U_1,\ldots,U_k]$. Thus,
  $t\in(\exists[U_1,\ldots,U_k] R)^{\Imc}$.

  \smallskip
  We now show that $\Jmc$ is a model of $\gamma(\A)$.\\
  Concerning axioms in~\eqref{eq:ob1} and~\eqref{eq:ob2} they are
  satified by construction. $\Jmc$ also satisfies axioms
  in~\eqref{eq:rei1} and in~\eqref{eq:rei2} due to~\eqref{eq:R}
  and~\eqref{eq:Q}, respectively, and the interpretation of $\xi$
  in~\eqref{int:xi}. Concerning axioms
  in~\eqref{eq:unique1}-\eqref{eq:unique2}, we set
  $Q_o^\Jmc = \{o^\Imc\}$, for each $o\in\Ob$, and
  $Q_t^\Jmc = \{o_1^\Imc\}$, for each tuple
  $t = \langle U_1\!:o_1,\ldots,U_n\!:\!o_n\rangle$ occurring in
  $\A$. We finally show that $\Jmc$ satisfies axiom~\eqref{eq:unique3}
  by considering, w.l.o.g., the case of binary tuples,
  $t = \langle U_1\!:\!o_1, U_2\!:\!o_2\rangle$. Then,
  $\pth{\tau(t)}{\{U_1\}}^\dag = Q_{U_1}$ and
  $\pth{\tau(t)}{\{U_2\}}^\dag = Q_{U_2}$. Assume that
  $o_1^\Jmc\in Q_t^\Jmc$ and that there are objects
  $d_1, d_2, d_3, d_4\in \Delta^\Jmc$ such that
  $(d_1,o_1^\Jmc), (d_2,o_1^\Jmc)\in Q_{U_1}^\Jmc$,
  $(d_1,d_3), (d_2,d_4)\in Q_{U_2}^\Jmc$ and
  $d_3, d_4\in Q_{o_2}^\Jmc$. We need to show that $d_1 = d_2$. We
  first notice that, since concepts $Q_o$ are interpreted as
  singleton, $d_3 = d_4 = o_2^\Jmc$. Furthermore, by~\eqref{eq:Q},
  $d_1 = \imath(t_1)$ and $d_2 = \imath(t_2)$, with $t_1 = \langle
  U_1\!:o_1^\Jmc, U_2\!:\!d_3\rangle$ and $t_2 = \langle
  U_1\!:o_1^\Jmc, U_2\!:\!d_4\rangle$ and thus $t_1 = t_2$. Since
  $\imath$ is injective, then $d_1 = d_2$.
\end{enumerate}

\smallskip

%%%%%%%%%%
($\Leftarrow$) Let $\Jmc=(\Delta^\Jmc,\cdot^\Jmc)$ be a model for the
knowledge base $\gamma(\KB)$.  Without loss of generality, we can
assume that $\Jmc$ is a \textit{forest model}. We then construct a model
$\Imc = (\Delta^\Imc, \cdot^\Imc, \rho, \imath, \ell_{R\!N_1},\ldots)$
for a \DLRpm knowledge base $\KB$. We set:
$\Delta^\Imc = \Delta^\Jmc$, $o^\Imc = o^\Jmc$ for all $o\in\Ob$, $C\!N^\Imc = (C\!N^\dag)^\Jmc$, for every
atomic concept $C\!N\in\Cmc$, while, for every ${R\!N}\in\Rmc$, we set:
%
\begin{multline}\label{eq:RN}
  R\!N^\Imc = \{t=\langle U_1\!:\!d_1,\ldots,U_n\!:\!d_n\rangle \in T_{\Delta^\Imc}(\tau(R\!N))\mid
  \exists d\in A_{R\!N}^\Jmc \text{ s.t. }\\ (d,t[U_i])\in (\pth{\tau(R\!N)}{\{U_i\}}^\dag)^\Jmc \text{ for } i=1,\ldots,n\}.
\end{multline}
%
We notice that~\eqref{eq:RN} defines a bijection between objects in
$\ALCQI$ reifying tuples and tuples themselves. Indeed, since $\Jmc$
satisfies $\gamma_{\textit{rel}}(R\!N)$, then, for every
$d\in A_{R\!N}^\Jmc$ there is a unique tuple
$\langle U_1\!:\!d_1,\ldots,U_n\!:\!d_n\rangle \in R\!N^\Imc$---we say
that $d$ \emph{generates}
$\langle U_1\!:\!d_1,\ldots,U_n\!:\!d_n\rangle$ and, in symbols,
$d\to \langle U_1\!:\!d_1,\ldots,U_n\!:\!d_n\rangle$. Furthermore,
since \Jmc is forest shaped, to each tuple whose components are not in
the ABox corresponds a unique $d$ that generates it. On the other
hands, since $\Jmc$ satisfies axiom~\eqref{eq:unique3}, then also for
tuples occurring in the ABox there a unique $d$ that generates them.
Thus, let $d\to \langle U_1\!:\!d_1,\ldots,U_n\!:\!d_n\rangle$, by
setting $\imath(\langle U_1\!:\!d_1,\ldots,U_n\!:\!d_n\rangle) = d$
and
%
\begin{multline}\label{eq:iota}
  \imath(\langle U_1\!:\!d_1,\ldots,U_n\!:\!d_n\rangle[\tau_i])=d_{\tau_i}, \text{ s.t. }\\
  (d,d_{\tau_i})\in(\pth{\{U_1,\ldots,U_n\}}{\tau_i}^\dag)^\Jmc,
\end{multline}
%
for all ${\tau_i\in\mathscr{T}_{\{U_1,\ldots,U_n\}}}$, then, the function $\imath$ is as required.\\
%
By setting
\begin{multline}\label{eq:lobj}
  \ell_{R\!N}(\langle U_1\!:\!d_1,\ldots,U_n\!:\!d_n\rangle) =
  d, \text{ s. t. }\\
  (\imath(\langle U_1\!:\!d_1,\ldots,U_n\!:\!d_n\rangle),d)\in
  Q_{R\!N}^\Jmc,
\end{multline}
%
then, by $\gamma_{\textit{lobj}}(R\!N)$, both $Q_{R\!N}$ and its inverse are
interpreted as a functional roles by \Jmc, thus
the function $\ell_{R\!N}$ is as required.\\
%
It is easy to show by structural induction that the following property holds:
\begin{align}\label{prop:RN}
\text{If } t\in R^\Imc \text{ then } \exists t'\in R\!N^\Imc \text{
  s.t. } 
t=t'[\tau(R)], \text{ for some } R\!N\in\R.
\end{align}
%
We now show that $\Imc$ is indeed a model of $\KB$. We first show that
$\Imc\models \Tmc$, i.e., 
$\Imc\models {C_1\sqsubseteq C_2}$ and
$\Imc\models R_1\sqsubseteq R_2$. As before, since
$\Jmc\models {C_1^\dag\sqsubseteq C_2^\dag}$ and
$\Jmc\models R_1^\dag\sqsubseteq R_2^\dag$, it is enough to show the
following:
  \begin{itemize}
  \item $d\in \Int{C} \text{ iff } d\in (C^\dag)^\Jmc$, for all \DLRpm\ concepts;
  \item $t\in \Int{R} \text{ iff } \imath(t)\in (R^\dag)^\Jmc$, for all \DLRpm\ relations.
  \end{itemize}
%
  The proof is by structural induction. The base cases are trivially
  true. Similarly for the boolean operators and global
  reification. We thus show only the following cases.\\
%
  Let $d\in(\lreif R\!N)^\Imc$. Then, $d=\ell_{R\!N}(t)$ with
  $t\in R\!N^\Imc$. By induction, $\imath(t)\in A_{R\!N}^\Jmc$ and, by
  $\gamma_{\textit{lobj}}({R\!N})$, there is a $d'\in\Delta^\Jmc$
  s.t. $(\imath(t),d')\in Q_{R\!N}^\Jmc$ and
  $d'\in (A_{R\!N}^l)^\Jmc$.  By~(\ref{eq:lobj}), $d=d'$ and
  thus, $d\in (\lreif R\!N)^{\dag\Jmc}$.
  \\
%
  Let $d\in(\exists^{\geq q}[{U_i}] R)^{\Imc}$. Then, there are
  different $t_1,\ldots,t_q\in R^\Imc$ s.t. $t_l[U_i]=d$, for all
  $l=1,\ldots,q$. For each $t_l$, by~(\ref{prop:RN}), there is a
  $t'_l\in R\!N^\Imc \text{ s.t. } t_l=t'_l[\tau(R)]$, for some
  $R\!N\in\R$, while, by induction, $\imath(t_l)\in R^{\dag \Jmc}$ and
  $\imath(t'_l)\in R\!N^{\dag\Jmc}$. Thus, $t'_l[U_i]=t_l[U_i]=d$ and,
  by~(\ref{eq:RN}),
  $(\imath(t'_l),d)\in (\pth{\tau(R\!N)}{\{U_i\}}^\dag)^\Jmc$ while,
  by~(\ref{eq:iota}),
  $(\imath(t'_l),\imath(t_l))\in
  (\pth{\tau(R\!N)}{\tau(R)})^{\dag\Jmc}$.
  Since \DLRpm allows only for knowledge bases with a projection
  signature graph being a multitree, then,
  $$\pth{\tau(R\!N)}{\{U_i\}}^\dag =
  \pth{\tau(R\!N)}{\tau(R)}^\dag\chain \pth{\tau(R)}{\{U_i\}}^\dag.$$
  Thus, $(\imath(t_l),d)\in (\pth{\tau(R)}{\{U_i\}}^\dag)^\Jmc$ and,
  since $\imath$ is injective, then, $\imath(t_l)\neq \imath(t_j)$
  when $l\neq j$. Thus,  $d\in(\exists^{\geq q}[{U_i}] R)^{\dag\Jmc}$.\\
%
  Let $d\in(\exists^{\leq q}[{U_i}] R)^{\Imc}$. By absurd, assume
  that $d\in(\exists^{>q}[{U_i}] R)^{\dag \Jmc}$, but then, as showed
  below, $d\in(\exists^{> q}[{U_i}] R)^{\Imc}$.
\\%
  Let $t\in(\selects{U_i}{C}{R})^\Imc$. Then, $t\in R^\Imc$ and
  $t[U_i]=d\in C^\Imc$. By induction, $\imath(t)\in R^{\dag\Jmc}$ and
  $d\in C^{\dag\Jmc}$. As before, by~(\ref{eq:RN}),~(\ref{eq:iota}) and~(\ref{prop:RN}),
  we can show that
  $(\imath(t),d)\in (\pth{\tau(R)}{\{U_i\}}^\dag)^\Jmc$ and, since
  $\pth{\tau(R)}{\{U_i\}}^{\dag}$ is functional, then
  $\imath(t)\in(\selects{U_i}{C}{R})^{\dag\Jmc}$.
  \\
%
  Let $t\in(\exists[U_1,\ldots,U_k] R)^{\Imc}$. Then, there is a tuple
  $t'\in R^\Imc$ s.t. $t'[U_1,\ldots,U_k]=t$ and, by induction,
  $\imath(t')\in R^{\dag\Jmc}$. As before,
  by~(\ref{eq:iota}) and (\ref{prop:RN}), we can show that
  $(\imath(t'),\imath(t))\in
  \pth{\tau(R)}{\{U_1,\ldots,U_k\}}^{\dag\Jmc}$ and thus
  $\imath(t)\in(\exists[U_1,\ldots,U_k] R)^{\dag\Jmc}$.\\
%
  All the other cases can be proved in a similar way. We now show the
  vice versa.

  \smallskip

  Let $d\in(\lreif R\!N)^{\dag\Jmc}$. Then, $d\in (A_{R\!N}^l)^\Jmc$
  and, by $\gamma_{\textit{lobj}}({R\!N})$, there is a
  $d'\in\Delta^\Jmc$ s.t. $(d',d)\in Q_{R\!N}^\Jmc$ and
  $d'\in A_{R\!N}^\Jmc$. By induction, $d'=\imath(t')$ with
  $t'\in {R\!N}^\Imc$ and thus, $(\imath(t'),d)\in Q_{R\!N}^\Jmc$ and,
  by~(\ref{eq:lobj}), $\ell_{R\!N}(t') = d$, i.e., $d\in(\lreif R\!N)^\Imc$.
  \\
%
  Let $d\in(\exists^{\geq q}[{U_i}] R)^{\dag \Jmc}$.
  Thus, there are different $d_1,\ldots,d_q\in\Delta^\Jmc$ s.t.
  $(d_l,d)\in (\pth{\tau(R)}{\{U_i\}}^\dag)^{\Jmc}$ and
  $d_l\in R^{\dag \Jmc}$, for $l=1,\ldots,q$. By induction, each
  $d_l=\imath(t_l)$ and $t_l\in R^\Imc$. Since $\imath$ is injective,
  then $t_l\neq t_j$ for all $l,j=1,\ldots,q$, $l\neq j$. We need to
  show that $t_l[U_i] = d$, for all
  $l=1,\ldots,q$. By~(\ref{prop:RN}), there is a
  $t'_l\in R\!N^\Imc \text{ s.t. } t_l=t'_l[\tau(R)], \text{ for some
  } R\!N\in\R$ and, by~(\ref{eq:iota}),
  $(\imath(t'_l),\imath(t_l))\in
  (\pth{\tau(R\!N)}{\tau(R)}^\dag)^\Jmc$.
  Since $(\imath(t_l) ,d) \in (\pth{\tau(R)}{\{U_i\}}^\dag)^\Jmc$ and
  $\textsc{path}_{\mathscr{T}}$ is functional in \DLRpm,
  then, $(\imath(t'_l) ,d) \in (\pth{\tau(R\!N)}{\{U_i\}}^\dag)^\Jmc$
  and, by~(\ref{eq:RN}), $t'_l[U_i]=t_l[U_i] =d$.\\
%
  Let $d\in(\exists^{\leq q}[{U_i}] R)^{\dag \Jmc}$. By absurd, assume
  that $d\in(\exists^{>q}[{U_i}] R)^{\Imc}$, but then, as proved
  before, $d\in(\exists^{> q}[{U_i}] R)^{\dag \Jmc}$.
\\
%
  Let $\imath(t)\in(\selects{U_i}{C}{R})^{\dag\Jmc}$. Then,
  $\imath(t)\in R^{\dag \Jmc}$ and, by induction, $t\in R^\Imc$. Let
  $t[U_i]=d$. We need to show that $d\in C^\Imc$. As before,
  by~(\ref{prop:RN}) and~(\ref{eq:iota}), we have that
  $(\imath(t),d)\in (\pth{\tau(R)}{\{U_i\}}^\dag)^\Jmc$. Then
  $d\in C ^{\dag\Jmc}$ and, by induction, $d\in C ^{\Imc}$.\\
%
  Let $\imath(t)\in(\exists[U_1,\ldots,U_k] R)^{\dag\Jmc}$. Then, there is
  $d\in\Delta^\Jmc$ s.t.
  \begin{equation}
    \label{eq:path-tuple}
     (d,\imath(t)) \in (\pth{\tau(R)}{\{U_1,\ldots,U_k\}}^\dag)^\Jmc
  \end{equation}
  and $d\in R^{\dag\Jmc}$.  By induction, $d=\imath(t')$ and
  $t'\in R^\Imc$. By~(\ref{prop:RN}), there is a tuple
  $t''\in R\!N^\Imc$ s.t. $t' = t''[\tau(R)]$ and, by~(\ref{eq:iota}),
  $(\imath(t''),\imath(t'))\in (\pth{\tau(R\!N)}{\tau(R)}^\dag)^\Jmc$
  and thus, by~\eqref{eq:path-tuple},
  $(\imath(t''),\imath(t))\in
  (\pth{\tau(R\!N)}{\{U_1,\ldots,U_k\}}^\dag)^\Jmc$ and thus
  $t = t''[\{U_1,\ldots,U_k\}]$. Since
  $\{U_1,\ldots,U_k\}\subseteq \tau(R) \subseteq \tau( R\!N)$, then,
  $t = t''[\{U_1,\ldots,U_k\}] = (t''[\tau(R)])[U_1,\ldots,U_k] = t'[U_1,\ldots,U_k]$, i.e.,
  $t\in(\exists[U_1,\ldots,U_k] R)^{\Imc}$.
%

To show that $\Imc\models\A$, notice that \Imc satisfies both concept
assertions and individual assertions by construction. We need to show
that \Imc satisfies also relation assertions. Let $R\!N(t)\in\A$, with
$t=\langle U_1\!:\!o_1,\ldots,U_n\!:\!o_n\rangle$,
then, since \Jmc satisfies $\gamma(\A)$, and in particular
axiom~\eqref{eq:rei1}, then there exists $d = \xi(t) \in
A_{R\!N}^\Jmc$. By~\eqref{eq:rei2}, $(d,o_i^\Jmc)\in
(\pth{\tau(R\!N)}{\{U_i\}}^\dag)^\Jmc$ and, by~\eqref{eq:RN},
$t^\Imc\in R\!N^\Imc$.
\hfill\qed

\end{proof}

As a direct consequence of the above theorem and the fact that \DLR is a sublanguage of \DLRpm, we have that

\begin{corollary}
  Reasoning in \DLRpm is an \ExpTime-complete problem.
\end{corollary}


%%%%%%%%%%%%%%%%%%%%%%%%%%%%%%%%%%%%%%%%%%%%%%%%%%%%%%%%%%%%%%%%%%%%%%
%%%%%%%%%%%%%%%%%%%%%%%%%%%%%%%%%%%%%%%%%%%%%%%%%%%%%%%%%%%%%%%%%%%%%%
\section{Implementation of a \DLRpm API}

%%%%%%%%%%%%%%%%%%%%%%%%%%%%%%%%%%%%%%%%%%%%%%%%%%%%%%%%%%%%%%%%%%%%%%
\section{Conclusions}

%%%%%%%%%%%%%%%%%%%%%%%%%%%%%%%%%%%%%%%%%%%%%%%%%%%%%%%%%%%%%%%%%%%%%%

\section{Acknowledgements}
We thank Alessandro Mosca for working with us on all the preliminary work necessary to understand how to get these technical results.

%%%%%%%%%%%%%%%%%%%%%%%%%%%%%%%%%%%%%%%%%%%%%%%%%%%%%%%%%%%%%%%%%%%%%%
%%%%%%%%%%%%%%%%%%%%%%%%%%%%%%%%%%%%%%%%%%%%%%%%%%%%%%%%%%%%%%%%%%%%%%

\bibliographystyle{named}
%%\bibliography{short-string,krdb}
\bibliography{biblio}

\end{document}

%%% Local Variables:
%%% mode: latex
%%% TeX-master: t
%%% save-place: t
%%% End:
